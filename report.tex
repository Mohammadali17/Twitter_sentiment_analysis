\documentclass[letterpaper, 12pt]{article}

%% Language and font encodings
\usepackage[english]{babel}
\usepackage[utf8x]{inputenc}
\usepackage[T1]{fontenc}
\usepackage{graphicx}
\usepackage{subfig}
\usepackage{blindtext}
\usepackage{multirow}
%% Sets page size and margins
\usepackage[letterpaper,top=2.4cm,bottom=2.4cm,left=3.6cm,right=3.6cm,marginparwidth=2.4cm]{geometry}

%% Useful packages
\usepackage{amsmath}
\usepackage{graphicx}
\usepackage[colorinlistoftodos]{todonotes}
\usepackage[colorlinks=true, allcolors=blue]{hyperref}

\title{Twitter Sentiment Analysis\\
Course Project Report\\
CS-583}
\author{
Mohammad Ghasemisharif, 
Mohammad Ali Bashiri, 
}
% \subtitle{Course Project Report }
% \subject{CS-583}
% \date{today}








\newcommand \bw {\mathbf{w}}
\newcommand \bW {\mathbf{W}}
\newcommand \bx {\mathbf{x}}
\newcommand\norm[1]{\left\lVert#1\right\rVert}
\setlength{\parindent}{0pt}
\begin{document}

\maketitle
\newpage

\section{Abstract}


Twitter plays an important role in US presidential election . A single tweet has the ability to shape political conversations and drive media coverage and affect the outcome of the election. Therefore, Sentiment analysis of tweets can provide interesting and valuable information, furthermore,  candidates can use the analysis in order to win an election. In this project, we analyze the tweets that form a discussion between presidential candidates Barack Obama and Mitt Romney at US 2012 presidential election. The goal is to predict to predict whether a given tweet is positive or negative. To train the model a set of training sentences is given where each sentence is annotated as positive or negative or neutral. In our model, the training -or test- data  is passed through several phases of pre-processing. Once we have a clean training data set, we train our models and apply the obtained model on test data set. In this project we use RapidMiner[1] for  pre-processing  step and classification. We explain the methods that we used and report the results on training dataset. 





\vspace*{10pt}
\section{Introduction}
Sentiment Analysis is the process of computationally determining whether a piece of writing is positive, negative or neutral. In other words, suppose a piece of writing is given, the task is to classify the piece of writing into three classes of positive, negative or neutral given a set of annotated writing as the training examples.   It’s also known as opinion mining, deriving the opinion or attitude of a speaker. This can be quite challenging due to complexity of training data and also because the tweets are most of the time messy in a sense that they are many spelling and grammar error in tweets, also a tweet could be from multiple languages and contains special characters and symbols such as hash-tags and URLs. So, compare to a general classification task, Twitter Sentiment Analysis, needs to clean the data very carefully in addition to choosing an appropriate classification method. Moreover, even with a proper pre-processing step, sometimes it is hard to find the actual intention of the tweet, due to the fact that tweets may contain different subtext or humor, which changes the meaning of the tweet and makes them ambiguous.  In this project, we tried to perform sentiment analysis on tweets of 2012 US presidential election from both Barack Obama and Mitt Romney. The tweets are classified to positive, negative, and neutral classes. We have used Rapid- Miner for pre-processing and building the supervised learning model.




\section{Tweet Analysis Technique}
\vspace*{10pt}

As we discussed earlier, our analysis consist of two parts, first, pre-processing the data and second, applying the classification methods. In this section, for each part, we explain the methods that we used in detail. Essentially, for pre-processing we describe all the methods to get a clean text and  for classification part, we describe all classifiers and that we used in our analysis.   


\vspace*{10pt}
\subsection{Pre-processing Methods}

Pre-processing the raw data takes up the majority share of the work, and often holds the key to good classification. In this project we have tried several pre-processing techniques, on the basis of the training data, of which we are stipulating the ones that gave the best results. In this section we describe each pre-processing method that we used in detail. Essentially, for pre-processing we describe case-converting, tokenization, filtering, stop-word removal, stemming, and building n-Gram model.\\


\vspace*{10pt}
\textbf{Case Conversion}\\

To get rid of any discrepancy that may arise due to inconsistency of representation of a word we convert all alphabets to their lower case representation. This conversion helps the model to treat the same words which appeared differently in different places the same.\\

\vspace*{10pt}
\textbf{Tokenization}\\


For a character sequence and a defined document, tokenization is the task of chopping the document up into pieces, called tokens , perhaps at the same time throwing away certain characters, such as punctuation. The Vector Space Model for information retrieval requires the document to be represented as a weight vector. The vector components are essentially the words or terms in the document. Therefore, the data is tokenized into words on the basis of spaces and special characters.\\

\vspace*{10pt}
\textbf{Filtering}\\


There are some elements in texts that are essentially uninformative and sometimes even confuses the outcome of a classifier. For example, some texts may contain URLs, which does not provide any valuable information for a better classification.   Filtering basically gets rid of uninformative elements of a text. In this project we used a filter to remove all instances that label is not provided for them. A filter that removes all the instances that has a missing attribute. We also add filters for removing emails, URLs, numbers, non-letter alphabets, and the words that has length less than 3 or more than 25.\\

\vspace*{10pt}
\textbf{Stop Word Removal}\\


There are words in any language that being used frequently but does not have any valuable information such as, to, a, such, etc. We used a list of common stop-words in English and removed all such words from the datasets.   



\vspace*{10pt}
\textbf{Stemming}\\

Stemming is to represent a word in terms of its roots. Almost all of the times,  there is no difference between the words like and likes. So, it is better to consider the root of each word, more precisely if we replace all the words with their roots then there is no redundant representation of a word in the model. \\

\vspace*{10pt}
\textbf{N-Gram Model}\\

N-Gram is a the most widely used statistical model to model a language. In this model, the assumption is that a test is a sequence of random variables where each random variable depends on the N previous variable assignments. 

\[
	P(w_1, w_2, \ldots, w_n) = P(w_1) P(w_2) P(w_3) \ldots P(w_n|w_1, \ldots, w_{n-1})
\]

The choice of N here depends on the task but usually N=3 works better than other assignemnt. Note that, this model handle the words that are not appeared in train but appeared in test using a smoothing method. \\



\vspace*{10pt}
\subsection{Classification Methods}
We select common text classification methods including Support Vector Machine, Logistic Regression,
Random Forest, and Naive Bayes as well as more complex methods such as Long short-term memory (LSTM) 
to perform the classification task. All of the methods are evaluated using 10-fold cross validation. 
Tuning parameters is done experimentally and thus for each classifier, we change the parameters 
and run the experiment multiple times to find the better precision and recall for individual and 
overal classes. 

We use SVM with C-SVC type and proceed the experiments with two different kernels: Linear and rbf 
kernels. We then adjust the parameters such as C, gamma, and cache size and compare the results. 
The value of C for Linear and rbf is set to 1000 and 500 respectively. Cache size is equally set 
to 100 and gamma, for rbf, is set to 0.1.  Logistic Regression is another text classifier that we use in our experiments. We use \emph{MeanImputation}
for handling missing values. We further explore Naive Bayes and Random Forest. Our Random Forest has 120 trees 
with max depth of 12 and uses gain ration and information gain. Lastly, our Naive Bayes is set to 
use laplace correction. 


% Please add the follow

Finally, we build a recurrent neural network using LSTM and train the data set to examine its 
performance. We implement the neural network using Keras. Unfortunately, the current version 
of Keras does not support performance metrics other than accuracy, thus we only included the 
accuracy result for LSTM. The network is composed of one layer convolutional with \emph{relu} activation
, one maxpooling with size of 4, and one layer LSTM. The activation function we use 
is \emph{softmax} and finally we use \emph{categorical cross entropy}
for the loss function. 


\section{Experimental Results and Evaluation}

In this project, we were given the training data set and the test set was given at the evaluation day. As we described we tried different data pre-processing methods and different classification method to obtain the best methods based on the performance on training data. 

Table~\ref{fig:res} shows the results of both 
classifiers for Obama and Romney tweets on training dataset.  As it was asked in the project definition we report precision (P), recall (R), and F-score (F) a metrics for each method.  These performance measures were obtained using 10-fold cross validation on the training data set. Table~\ref{fig:res} was partially constructed on presentation 1 \& 2.  Note that for the given data set the performance metrics for Romney negative class is always higher than the positive class since the data set is imbalance and the data distribution is skewed over negative class. 

       
more data for negative class compared to the positive class.


\begin{table}[t]
        \centering
        \caption{Classification methods results}
        \label{fig:res}
        \resizebox{\textwidth}{!}{\begin{tabular}{l|l|l|l|l|l|l|l|l|l|l|l|l|l|l|}
        \cline{2-15}
        & \multicolumn{7}{c|}{Obama} & \multicolumn{7}{c|}{Romney} \\ \cline{2-15} 
        & \multicolumn{3}{c|}{Positive} & \multicolumn{3}{c|}{Negative} & \multirow{2}{*}{Accuracy} & \multicolumn{3}{c|}{Positive} & \multicolumn{3}{c|}{Negative} & \multirow{2}{*}{Accuracy} \\ \cline{2-7} \cline{9-14}
        & Precision & Recall & F-Score & Precision & Recall & F-Score &  & Precision & Recall & F-Score & Precision & Recall & F-Score &  \\ \hline
       \multicolumn{1}{|l|}{SVM Linear} & 68.46 & 55.21 &  & 65.92 & 52.29 &  & 59.41 & 62.26 & 32.84 &  & 60.26 & 86.28 &  & 58.73 \\ \hline
       \multicolumn{1}{|l|}{SVM rbf} & 67.54 & 59.98 &  & 63.97 & 60.76 &  & 60.88 & 33.67 & 63.18 &  & 60.77 & 85.07 &  & 58.91 \\ \hline
       \multicolumn{1}{|l|}{Logistic Regression} & 58.49 & 57.06 &  & 64.44 & 37.20 &  & 54.39 & 45.43 & 45.77 &  & 66.72 & 58.62 &  & 53.70 \\ \hline
       \multicolumn{1}{|l|}{Naive Bayes} & 48.92 & 49.73 &  & 60.48 & 35.92 &  & 49.63 & 32.48 & 56.37 &  & 59.70 & 65.17 &  & 49.81 \\ \hline
       \multicolumn{1}{|l|}{Random Forest} &  &  &  &  &  &  &  &  &  &  &  &  &  &  \\ \hline
       \multicolumn{1}{|l|}{LSTM} &  &  &  &  &  &  & 55.73 &  &  &  &  &  &  & 54.12 \\ \hline  
        \end{tabular}}
        \end{table}


As we can see in table~\ref{fig:res} non-linear Support Vector Machines with RBF kernel has the best performance on the training data set. So, in the final demo we used this method to predict the classes of test data set.  








\section{Bibliography}


 [1] Rapidminer - https://rapidminer.com.






\end{document}